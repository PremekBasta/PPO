\chapter{Our work - Contribution}

"From the perspective of game theory, we are interested
in n-person games in which the players have a shared or joint
utility function. In other words, any outcome of the game has
equal value for all players. Assuming the game is fully co-
operative in this sense, many of the interesting problems in
cooperative game theory (such as coalition formation and ne-
gotiation) disappear. Rather it becomes more like a standard
(one-player) decision problem, where the collection of n play-
ers can be viewed as a single player trying to optimize its be-
havior against nature." 

"Solutions to the coordination problem can be divided into
three general classes, those based on communication, those
based on convention and those based on learning"

Convention probably does not make sense as we have ad hoc partner
Craig Boutilie 1996

\section{Our definition(s?) of robustness}
Probably just average of pair results (non diagonal in case of same sets).
Maybe percentage of pairs who surpassed some threshold reward?

\section{Population construction}

\subsection{SP agents initialization}
\textbf{One agent is not enough?}

\subsection{population partner sampling during training}
\textbf{See if playing with whole population at once differs from one random partner for episode}

\subsection{Final agent training}

\section{Diverzification}
\textbf{maximize kl divergence among population partners policies}

\subsection{Population policies difference rewards augmentation}

\subsection{Population policies difference loss}
