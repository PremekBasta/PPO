\chapter{Related work}

A common approach to two-player games is to train an agent by confronting it with a set of other AI agents.
This has been shown to perform impressively well against human experts in several complex games such as Dota (\cite{DotaOpenFive}) or Starcraft (\cite{Starcraft}).
The authors of the overcooked environment (\cite{carroll2020utility}) believe that the distributional shift from AI training to human evaluation is successful due to the fact that the nature of this environment is strongly competitive.
This is illustrated by the canonical case of a two-player zero-sum game.
If humans take a branch in the search tree that is inadvertently suboptimal while in the role of minimizer, this only increases the outcome for the maximizer.

However, the situation is not so ideal in the case of common payoff games such as overcooked environment.
If a self-play trained AI agent is paired with another sub-optimal partner, the result can be abysmal. 
In this case, both agents play a maximizing role in the search tree. 
If the self-play AI agent expects it's sub-optimal partner to be the same agent, it can choose branches that lead to maximum joint payoff.
However, since the sub-optimal partner may behave sub-optimally, it may happen that it mistakenly chooses some worse branches that are unforeseen by the self-play agent, which may lead to a significantly worse result.
However, since the common payoff is shared between the two agents, this is no longer an advantage for the self-play maximizing agent.
Rather, it leads to failure for both agents.

\section{Human cooperation}
Most previous work in this area has focused on one of two types of coordination. 
The first is coordination between a human and an AI partner. 
And second, focusing solely on the fully AI-driven pair.


\par

While perfect AI-human coordination is generally a more desirable goal to achieve in all sorts of domains, it will not be our main focus.  
Several previous scientific papers have addressed this issue. A particularly noteworthy contribution is the paper 
On the Utility of Learning about Humans for Human-AI Coordination (\cite{carroll2020utility}). 
They collected several human-human episodes and incorporated these experiences into training.
This human data was used to create human-based models using behavior cloning and Generative Adversarial Imitation Learning (\cite{Ho2016GenerativeAI}) and then incorporated into training.
One of the important conclusions was that when self-play and population-based agents were paired with human models, the overall results were substantially worse than when paired with agents designed to play with human models.

\section{AI-AI cooperation}\label{aiaicooperation}
We have already mentioned the problem when a self-play agent is paired with a suboptimal (e.g. human) partner.
However, a similar problem can also occur when two different agents trained in self-play mode are paired together.
A common approach used in competitive settings is to introduce diversity by exposing the trained agent to a diverse set of partners during the training phase.
There are several popular methods for partner selection during training.

\subsection{Self-play}\label{selfplayMethod}
The self-play method has already been mentioned, and it is arguably the simplest and most straightforward method for partner sampling.
As the name indicates, the idea of this approach is that the agent currently being learned is paired with the copy of itself during training.
This method doesn't introduce any kind of diversity into the learning process, as it only learns from its own current policy.

\subsection{Partner Sampling}
Partner sampling is an extension of the previous method. 
However, instead of playing only with the current policy, the partner is sampled from previously periodically saved policies of this given training period.
The diversity in this method comes from the theoretical point of view that previous policies correspond to different behaviors.

\subsection{Population-Based Training}
The population-based training method is based on an evolutionary algorithm that focuses on hyperparameter training and model selection.
The population consists of agents parameterized by a neural network and trained by a DRL algorithm.
During each iteration, agents are drawn from the population and trained using the collected trajectories.
Pair-wise results are recorded, and at the end of the iteration, the worst agents are replaced by a copy of the best agents with their hyperparameters mutated.

\subsection{Pre-trained Partners}
Finally, a often employed method is to employ pre-trained partners.
Diversity is expected here due to the fact that different runs of reinforcement learning algorithms often yield different agent behaviors.


\subsection{Result}\label{offDiagonalReport}
The problem with fully cooperative games is that when agents are trained together, they tend to exploit the common knowledge they have learned during training, which often makes them unable to cooperate with unseen agents.
It has been shown (\cite{10.1007/978-3-030-63823-8_46}) that this is true for the first three methods mentioned, 
where, when performing a cross-play evaluation of the set of different agents obtained by the same method, 
only those pairs of agents that were explicitly trained together performed well, while the remaining pairs failed.

{\color{blue} Nemam si tady od nich z toho clanku pujcit primo ty obrazky ukazujici ten SP off-diagonal failure, abych se na to pak mohl vizualne odkazat ve svych experimentech? 
Nebo krast obrazky se nedela?}
The authors conclude with results showing that incorporating different pre-trained agents for training robust agents was significantly more successful than other methods.


\section{Problem of robustness}
Another important issue related to common payoff is evaluating the performance of trained agents.
As discussed earlier, we cannot rely on training performance alone, no matter what metric is chosen, because training sets of agents exploit shared knowledge and thus make training performance appear excellent.
As noble as it sounds, it is quite complicated to define which agent behavior is actually robust.
Obviously, it's a desirable goal to create a trained agent that can cooperate well with all possible behaviors of its partner.
However, since it is difficult to come up with a diverse set of agents for partner sampling during training, the same is true for evaluation agents.

\subsection{Average performance}

The first obvious choice of evaluation metric could be to look at the average evaluation performance over the evaluation set of agents.
It is questionable, though, whether this answers the question of robustness.
In one scenario, we could theoretically achieve a result where our agent fully fails to cooperate with half of the evaluation agents, while performing excellently with the other half.
In another case, we could achieve some kind of intermediate performance with the whole set of evaluators, where cooperation does not fail completely, but performance is also nowhere near the optimal values.
In both cases, the average evaluation performance reaches similar values. 

\subsection{Threshold performance}

Another possible approach for the given layout could be to define our own threshold, which could be practically equivalent to a certain number of delivered soups.
The evaluation metric could then simply be expressed in terms of the number of evaluation agents that managed to reach such a threshold when paired with our trained agent.

\subsection{Edge case testing}
The problem with previous metrics is that they are reasonable only in a situation where the agent's evaluation set is complex enough in terms of behavioral diversity, which also implies a condition on the sufficient size of such a set. 
However, creating an evaluation set that satisfies this condition is practically impossible.

Instead of trying to construct such an evaluation set and trying to evaluate whole episodes in an effort to test cooperation on the run, we could break the interesting cooperation-challenging situation into a number of separate tests (\cite{knott2021evaluating}).
Suddenly, our thinking about robustness shifts from looking at cooperation from the point of view of the outcome of the entire episode to just separate small modular situations where the desired correct behavior of both agents can be described.
This approach is inspired by unit testing in software development, where even if the program behaves correctly in the vast majority of cases, it can still produce undesirable responses in edge cases.


Similar to unit testing, we can define a wide set of situations (e.g., instances of overcooked environment states) and pair them with a set of acceptable consequent behaviors of agents that can be labeled as robust response.
Evaluation using such a metric can then be expressed as the number of tests that have passed.

However, there are several challenges with this metric as these edge cases require manual design, in many scenarios it is also difficult to know with certainty what the correct expected behavior is, and finally, similar to unit testing in software development, there are never enough defined edge cases to cover all possible situations.