\documentclass{report}
\usepackage{hyperref}
\begin{document}

\chapter{General MARL/PPO}

\section{The Surprising Effectiveness of MAPPO in Cooperative, Multi-Agent Games}
\url{https://www.researchgate.net/publication/349727671_The_Surprising_Effectiveness_of_MAPPO_in_Cooperative_Multi-Agent_Games}

\begin{list}{}{}
    \item PPO sample efficiency
    \item 1 GPU desktop, 1 Multicore CPU for training
    \item centralized value function - global state s insted of local o
    \item environments: Particle world environment
    \item PPO used be seen as sample less efficient, hence for academic purposes MADDPG anv value-decomposed Q-learning
    \item Minimal hyperparameter tuning and no domain specification
    \item Decentralized learning each agent its own policy, suffer from non-stationary transitions
    \item two lines of research - CTDE (this) and value decomposition
    \item in single agent PG advantage normalization is crucial
    \item considered implementation details - input norm, value clipping, orthogonal init, gradient clip - all helpful and included
    \item another - discretization action space for PPO to avoid bad local minima in continuous, layer normalization
    \item MLP vs Recurrent
    \item 5 implementation details: \newline
        Value norm: running average over value estimates, value network regress to normalized target values (Pop art technique) \newline
        Agent-specific global state: concate of all o\_i as input to critic  \newline
        (agent specific global cannot be used in QMix, which uses single mixer network common to all agents) \newline
        Training Data Usage: importance sampling to perform off-policy correction ?? \newline
            multiple epochs may suffer from non stationarity -> 15 to 5 epochs (easy to hard) \newline
            No mini-batches -> more data to estimate gradients -> imporved practical performance \newline
        Action masking: unavailable actions when computing action probabilities - both forward and backward \newline
        Death masking: zero states with agent ID as value input for dead agents        

\end{list}

\chapter{Overcooked related}
\url{https://github.com/HumanCompatibleAI/overcooked_ai}

\section{On the Utility of Learning about Humans
for Human-AI Coordination}
\url{https://arxiv.org/abs/1910.05789}
\begin{list}{}{}
    \item agents assume their partner to be optimal or similar fail to be understood by humans
    \item gains come from having agent adapt to human's gameplay
    \item effective way to tackle two-player games is train agent with set of other AI agents, often past versions
    \item collaboration is fundamentally different from competition (we need to go beyond self-play to account for human behavior)
    \item incorporating human data or models into training leads to significant improvements (behavior cloning model)
    \item Population Based Training is online evolutionary alg, adapts training hyperparameters and perform model selection
    agents, whose policies are parametrized by NN and trained with DRL alg. each PBT iteration pair of agents are drawn, trained for number of steps and have performance recorded
    at end of PBT iteration, worst performing agents are replaced with copies of best and parameters mutated
    \item human behavior cloning peformed better than with Generative Aversarial Imitation Learning (GAIL)
    \item PBT better than PPO self-play because they are trained to be good with population coordination
    \item Agents designed for humans. Start with one learned human BC as part of environment dynamic and train single agent PPO.
    \item start with ppo self-play and continue with training with human model
    \item planning alg A*
    \item two human behavior cloning models Hproxy used for evaluation and PPOBC learned against learned human models
    \item best result self-play with self-play
    \item for human interaction was best PPOBC with HProxy...PPOBC is overall preferable
    \item PPOBC outperformes human-human performance
    \item SP agents became very specialized and so suffered from distributional shift when paired with human models
    \item future work - better human models, biasing population based training towards humans
    \item READ AGAIN if interested
\end{list}

\section{PantheonRL:
A MARL Library for Dynamic Training Interactions}
\url{https://iliad.stanford.edu/pdfs/publications/sarkar2022pantheonrl.pdf}
\begin{list}{}{}
    \item PantheonRL new software package for marl dynamic
    \item Combination of PettingZoo and RLLIB - customziation of agents
    \item prioritizes modularity of agent objects - each has separate replay buffer, own learning alg, role
    \item (other powerfull DRL library - StableBaselines3)
    \item The modularity of the agent policies com-
    bined with the inheritance of StableBaselines3 capabilities
    together give users a flexible and powerful library for ex-
    perimenting with complex multiagent interactions
    \item 
\end{list}





\section{Investigating Partner Diversification
Methods in Cooperative Multi-agent
Deep Reinforcement Learning}
\url{https://www.rujikorn.com/files/papers/diversity_ICONIP2020.pdf}
\begin{list}{}{}
    \item PBT have diversity problem -> PBT agents are not more robust than self-play agents and aren't better with humans
    \item creating diversity by generating pre-trained partners is simple but effective
    \item (partner sampling - playing with uniformly sampled past versions of partner - lacks diversity, past versions have similar behavior)
    \item (population-base training, pre-trained partners)
    \item testing self-play and cross-play with these agent types (SP, SPpast, PBT, PTseeds, PTdiverse)
    \item PTdiverse(hyperparameters) and PTseeds come from self-play
    \item ref: ustesen, N., Torrado, R.R., Bontrager, P., Khalifa, A., Togelius, J., Risi, S.: Illu-
    minating generalization in deep reinforcement learning through procedural level
    generation. arXiv preprint arXiv:1806.10729 (2018)
    \item IDEA to do: combine pretrained agents with maximum cross entropy? Encorporate maxium cross entropy into ppo??
    \item LOVED THIS ARTICLE for it's simplicity
    
    
\end{list}


\section{Evaluating the Robustness of Collaborative Agents}
\url{https://arxiv.org/abs/2101.05507}
\begin{list}{}{}
    \item how test robustnes if cannot rely on validation reward metric
    \item unit testing (from software engineering) - edge cases, eg. where soup was cooked but not delivered
    \item incorporating Theory of mind to human model 
    \item human modal diversity by using population of human models
    \item state diversity - init from states visited in human-human gameplay
    \item test suite provides significant insight into robustness that is not correlated with average validation reward
    \item "improved robustness as measured by test suite, but decrease in average validation reward"
    \item simple ML metrics are insufficient to capture performance and we must evaluate results base on human judgement
    \item domain randomization - some aspects of env are randomized - behavior can vary significantly based on small randomization of "irrelevant" factors
    \item Theory of mind??? - each step agent decides what task/strategy to do (eg. deliver soup), then choose low-lever action (motion) to persue this goal.
    \item Population of BC models, ToM models, or mixture of two
    \item recurrent networks for all deep rl training procedures
    \item "once the trained policy has
    found a good strategy for getting reward, it is not incentivized to explore other areas of the state space"
    \item sampling initial state from human-human data (diverse starts)
    \item Future work - how evaluate robustness in cases of ambiguous behavior
    \item Or evaluating of proposals that populations with BC had positive effect, but negative for ToM
    \item Meta learning for any kind of game layout, not just those prefabricated (online learning)
\end{list}

\section{Interaction Flexibility in Artificial Agents Teaming with Humans}
\url{https://www.researchgate.net/publication/351533529_Interaction_Flexibility_in_Artificial_Agents_Teaming_with_Humans}
\begin{list}{}{}
    \item too psychological, empirical studies of real people experience when playing with self-play / human BC agents
\end{list}

\section{Maximum Entropy Population-Based Training for Zero-Shot Human-AI Coordination}
\url{https://arxiv.org/abs/2112.11701}
\begin{list}{}{}
    \item "problem of training a Reinforcement Learning (RL) agent that is
    collaborative with humans without using any human data"
    \item "To mitigate this distributional shift, we propose Maximum Entropy Population-
    based training (MEP). In MEP, agents in the population are trained with our
    derived Population Entropy bonus to promote both pairwise diversity between
    agents and individual diversity of agents themselves, and a common best agent is"
    \item Comparing MEP with PPO self-play, PBT, Trajectory diversity and DIctitious CO-play
    \item diverse and distringuishable behaviors between all agent pairs utilizes average KL divergence between all agent poairs
    \item each agent in population is rewarded to maximize centralized population entropy.
    \item we train best response agent by pairing it with with agent sampled by difficulty to collaborate with (prioritizing)
    \item each agent has maximum entropy bonus (to reward) to encourage policy itself to be exploratory
    \item Popuplation diversity = average entropy of each agent + average KL divergence of pairs
    \item Populatino entropy = bounded and efficient surrogate for optimization = entropy of mean policies of population
    \item PE is lower bound for PD
    \item Not using PPO, but custom Entropy loss functions
    \item Population entropy (effective linear pairwise kl divergence) as part of objective
    \item MEP shares intuition with domain randomization... (MEP can be seen as domain randomization technique over partners policies)
    \item (TrajeDi = encourages trajectories diffference between agents - Jensen-SHannon divergence between policies)
    \item (Diversity-Inducing POlicy Gradient = formulated for single agent setting)
    \item Bridges maximum entropy RL and PBT... entropy maximization for achieving robustness
    \item Combining MEP with other MARL algorithms could be Future work
    \item Idea: apply r=r+alpha * population entropy?
    \item Idea: ("Maximum entropy approach adds the dense entropy to the reward for each time step, while entropy regularization adds the mean entropy to the surrogate objective") 
        "Note that
        Entropy regularization is not, in general, equivalent to the maximum entropy objective, which not only optimizes for
        a policy with maximum entropy, but also optimizes the policy itself to visit states where it has high
        entropy. Put another way, the maximum entropy objective optimizes the expectation of the entropy
        with respect to the policy’s state distribution, while entropy regularization only optimizes the policy
        entropy at the states that are visited, without actually trying to modify the policy itself to visit high-
        entropy state"
        \url{https://garage.readthedocs.io/en/latest/user/algo_ppo.html}
    \item 
    \item INTERESTING article
\end{list}

\section{Assisting Unknown Teammates in Unknown Tasks: Ad Hoc Teamwork under Partial Observability}
\url{https://arxiv.org/abs/2201.03538}
\begin{list}{}{}
    \item Ad hoc teamwork under partial observability (ATPO)
    \item unknown teammates performing unknown task withou pre-coordination protocol
    \item ad hoc teamwork has three parts - task identification, teammate identification and planning
    \item 
\end{list}


What is Zero-shot coordination?? - studies how independently trained agents may interact with another on first-attempt
\section{}
\url{}
\begin{list}{}{}
    \item
\end{list}


\section{Work progress}
\subsection*{7.11.2022}
\begin{list}{}{}
    \item manipulace set agent id pro jednotliva env, jakym zpusobem se zpracovavaji odesilane obs a prijimane actions
        je potreba toto pak jakkoliv resit na urovni stable baselines strev?
        vypada to, ze ne ... Na strane RUNNER: v $obs[0]$ je vzdy obs pro self.model a v $obs[1]$ je vzdy obs pro self.other agent model
        Runner vzdy vytvori joint actions jako (self actions, other agent actions) a posle je do env
        Env je zpracuje, podle agent id budto necha, nebo spravne prohodi do joint actions
        Env s joint actions provede step, ziska obs a opet spravne dle agent idx budto necha nebo prohodi a vraci (obs0, obs1),
            A opet jsme na zacatku ...  Runner se muze spolehnout ze v $obs[0]$ ma obs toho trenovaneho modelu self.policy a v $obs[1]$ ma obs pro other agent model (de facto embedded into environment)

    \item kolik env steps provadi Runner? nelze nastavit pevne jedna epizoda == 400? => parametr v PPO()
    \item proc overcooked env reset musi resetovat mlam? Zatim zakomentovano, TODO: proverit
    \item learning rate zatim neni annealovany
    \item zakomentovany nektere metriky, kterym zatim nerozumim nebo se mi nemeni (clip fraction, clip range, learning rate)
\end{list}

\subsection*{15.11.2022}
\begin{list}{}{}
    \item Struktura inspiravana projektem max population entropy, take pouzito stable baselines
    \item stable baselines posledni oficialni podpora TF1, neoficialni podpora TF2
    \item nove doporucene Stable baselines3 ktere pouziva pytorch
    \item S pytorch moc neumim, ale pozmenit reward a pridat do loss rozdil oproti populaci asi neni problem, takze zalezi jak moc velke zmeny ocekavam
    \item Adaptace na SB3 docela jednoducha, vektorizovane prostredi
    \item Other agent jako soucast "single agent" prostredi
    
    Technicke problemy:
    \item Prostredi se vzdy resetuje do stejneho stavu
    \item Struktura CNN, MLP vs reprezentace stavu lossless, featurize state mdp
    \item RNN, Frame stacking, nebo staci reakcni agent?
    \item Jak vyhodnocovat agenty deterministicky arg max dopadne kazda epizoda naprosto stejne, nedeterministicky teoreticky nedostavam nejlepsi vysledek, nebo vyhodnocovany agent deterministicky a other agent nedeterministicky? Zatim nevim jak to resi v ostatnich projektech
    \item Annealing entropy coeficient?, annealing sparse r coef
    \item Napevno 5M env steps staci? Da se o tom rict "with little loss of generality"? Nebo tohle muze byt zkreslujici pro vysledek mych experimentu?
    \item 5 zajimavych map
    \item diff bonus: k dispozici $log_prob$ pro danou akci, $e^log_prob = prob, diff =? min_pop(p(a) - a(a))^2 =? (mean(pop(a) - a(a)))^2$
    \item Konkretni plan: Zacit s jednou mapou, natrenovat si 10 self-play agentu (pripadne dalsich 10 s ruznymi hyper-parameters), ktere si necham pro nezavisle testovani,
        Pak zacit trenovat tim pridavacim zpusobem populaci a kazdeho noveho jedince otestovat vuci vsem 10 testovacim. Teorie je takova ze s kazdym pozdeji pridanym agentem ma byt vysledek lepsi vuci test agentum.
        Zacit s diff bonusem = 0 pro porovnani a pak zkouset bonus zvysovat (0.1).
        To same pro dalsi mapu a zkusit najit nejakou zajimavou hodnotu pro diff 

\end{list}

\subsection*{2.12.2022}
\begin{list}{}{}
    \item Zatim se nedari dosahnout SP off-diagonal failure
    \item Zkousim vymyslet stejnou hodnotu entropy coef napric vsemi expermienty. Pro 0.01 parkrat nezkonvergovalo.
        Zkousim, jestli nepomuze prodlouzit dobu ziskavani castecnych odmen.
    \item Zkousim jak moc by slo snizit celkovy pocet kroku trenovani, aby to bylo jeste stabilni a zaroven aby to netrvalo tak dlouho  
    \item Chce to zrefaktorovat a rozmyslet strukturu evaluace/treninkovych zpusobu a mnozin/ vizualizace, aby vse fungovalo obecne
    \item Z dnesni spolecne konzultace s Martinem a Petou -> Peta se planuje zabyvat komplexnejsimi mapami nebo komplexnejsim prostredim (pr. vice receptu)
        Zatim se mi nedari moc nasimulovat ten problem s robustnosti, velka cast agentu se mi zatim jevi jako relativne kompatibilni.
        Komplexnejsi prostredi by to treba mohlo pomoct rozbit.
        Zaroven by tam pak chybelo porovnani vuci jinym jiz existujicim pracim, protoze vsechny prace docela shodne pracuji jen s pevnou mnozinou 5 pevnych map 
    \item TODO: Potreba rozmyslet metriku pro rozdil vuci populaci, MSE se mi zda ze je stejne jako rozdil vuci prumerne distribuce populace, coz mi Martin rozmlouval
        Mozna nejakou prob dist. metriku, KLL divergenci?  

\end{list}

\subsection*{4.12.2022}
\begin{list}{}{}
    \item Mozna souvislost mezi off-diagonal faileru a entropy coeficient
        Za pozornost stoji SP\_RS\_E0\_IMPORTANT\_Entropy.png,  kde prvni polovina agentu je natrenovana s ent coef $==$ 0
        zatimco druha polovina s ent coef == 0.02
        Provnanim leveho horniho ctverce a praveho dolniho ctverce vidime velky rozdil ve vysledcich
    \item Zkousim ted rychly experiment (1M stepů) s coef 0, 0.01, 0.02 a uvidime
    \item Pro nizke ent coef se casto stava ze chovani nezkonverguje k necemu rozumnemu - 
        pridal jsem early stopping kdyz agent nezvladne v prvnich N krocich uvarit jedinou polevku
    \item prechozi verze evaluatoru vyuzivajici MDP evaluator se mi zdala ze fungovala zvlastne (pr. SP agenti obcas dopadli se sebou samymi katastrofalne)
        Implementoval jsem evaluator, ktery se chova co nejvic podobne simulacim pri trenovani, jen vzdy preferuji deterministicke akce
        Pripada mi, ze vypada lepe - diagonala neselhava pro SP
        TODO: zjistit proc se lisi od originalniho evaluatoru... Neni tam chyba?
    \item TODO: refaktoring struktury a nazvu porovnani dvou mnozin
    \item TODO: device cuda funguje daleko pomaleji nez cpu
        Nedari se mi pouzivat cpu pro predikce pro training samples a cuda pro samotny learn


\end{list}

\subsection*{6.12.2022}
\begin{list}{}{}
    \item Implementoval jsem early stopping evaluaci, kdy koncim SP trenink, kdyz evaluace konci nad threshold hodnotou
        Tim padem minimalizuju riziko ze na diagonale budu mit spatne hodnoty
    \item Pripada mi, ze s early stopping nemusim nutne resit nejakou pevnou hodnotu entropy coeficientu.
        Snad by mohlo stacit zacit s vysokou entropii, annealovat ji k nule a hledat early stopping reseni
    \item Kdyz jsem pouzil pri treninku vic random stavy (p\_rnd v start\_state\_fn), nez pak pri evaluaci, tak to vypadalo ze se zlepsila situace na diagonale
        Tim padem se by se snad mohla "zhorsit" off diagonal
    \item Naopak pri mene random trenink stavu nez pri evaluaci, tak to vedlo k vic random vysledkum
    \item TODO: Zrejme ma smysl pri treninku vzit co nejvic random stavy. Ma smysl pri evaluaci random snizit?
    \item TODO: Stale neni vyresene jak vyhodnocovat. 
        Napady pro SP vyhodnoceni: 
            Pevny threshold > True or False a spocitat pocet
            Jako threshold pouzit na kazde pozici hodnotu z diagonaly  
            Jak se postavit k porovnani mych vysledku vs vysledkum v clanku?

        Napady pro vyhodnoceni populace (VS SP mnozina):
            Prumer jedince z populace vuci vsem SP agentum
            Zkouset zase pocet nad nejaky threshold?

    \item TODO: Nejake napady, jak prostredi ztizit? Jestli se tim nedosahne vetsi off-diagonal SP failure.
        Co pak s porovnatelnosti vuci jinym clankum?

\end{list}

\subsection*{Konzultace 6.12.2022}
\begin{list}{}{}
    \item Mozna se ten off-diagonal problem neprojevuje kvuli tomu, ze pracuji s jednodussim prostredim
        Tj. nevyuzivam loss\_less state a tim padem ani konvolucni site
    \item Martin si mysli jestli kvuli tomu jinemu nastaveni prostredi nekonverguji vsechny agenti ke stejnemu/podobnemu chovani
    \item Mozne dalsi kroky: Vyzkouset, jestli bude situace stejna na ostatnich mapach. Mozna jsou ostatni tezsi.
    \item Vyzkouset zmenit na loss\_less a pridat konvolucni vrstvy viz (https://arxiv.org/abs/1910.05789v2)

\end{list}

\subsection*{Konzultace 13.1.2023}
\begin{list}{}{}
    \item Loss\_less funguje, chyba v normalizaci
    \item hyperparametry dle clanku moc nefunguje
    \item nalezl jsem parametry s vahou critic aspon 0.1 ktere mi funguji pro vsechny mapy... random zkousenim cca 100 moznych kombinaci hyperparametru
    \item Obcas selze vypada ze pouze pro posledni mapu - mam predcasne ukonceni po 2.5M (cca 12 minut), mozna by mohla byt situace jina s "chytrymi pomocniky"
    \item cramped rooom a asynchronous advantages moc nemaji off-diagonal problem, plus vetsi randomizace prostredi to jeste zhorsuje, nebo stezuje uceni
    \item Pripada mi ze entropy coef 0.1 byl moc silny, ale pouze domnenka, nemam cas se tomu vic venovat
    \item Plan pokracovat na trech mapach Pro kazdou predtrenovat 3x10 vyhodnocovaci skupiny
    \item Zacit s populacnim trenovanim - dve myslenky, pridat inverse annealing spolecne reward
    \item Obcas ma vysledny model na konci eval 0 prestoze v prubehu na tom nebyl spatne - tj. budu asi brat nejlepsi model z prubehu. 
        V pripade populace rozmyslet eval, vuci vsem v populaci, aby to nebylo zkreslene
    \item Zacit s pop diff rewardem, MSE na prop vectoru asi neni dobry napad, tak spis KL, po konzultaci neresim poradi vstupu do KL\_DIV, proste jedno vybereme 
\end{list}

\subsection{Summary 25.1.2023}
\begin{list}{}{}
    \item Predelal jsem args na Namespace argparses, aby slo dobre volat z prikazove radky
    \item Prosel jsem si nezavisle conda instalaci bez instalace samotneho stable-baselines3 a overcooked, protoze existuji jako clone v projektu. Momentalne zjistuju jak vytvorit shell script ktery zvladne aktivat conda env a pustit entrypoint
            Zbyva asi vyresit sys.path.append cesty k overcooked\_ai\_py a stable\_baselines3
    \item TODO: zkontrolovat co se vlastne pocita v tom mem evaluate. Pripada mi ze to nevraci hodnoty c*20 / \#parallel envs
    \item Pro populacni trenovani jsem opustil myslenku jednoho pevneho soupere pro celou trenovaci epochu. Namisto toho rozdelim populaci mezi 30 paralelnich prostredi vuci kterym trenuju. 
        Tim mam zajisteno, ze to pri treninku stale vidi celou populaci. Tento pristup hodne zpomaluje trenovani, protoze najednou musim v kazdem stepu volat vickrat NN.
        Pro vyreseni nedelitelnosti na zacatku treninkove epochy populaci shuffluju
    \item Zkousel jsem pri populaci dva ruzne pristupy vzhledem k shared rewards. Prvni je nechat shared rewards tak jak jsou, toto mi nepripada intuitivni, nebot muze stacit aby partner vedet jak varit polevku a pak se ani nemusim snazit/nemusim na tom mit podil a presto mam odmenu.
        Proto jdu druhou cestu, kdy inverse annealuju shared rewards opacne k tem partial rewards. Tedy intuice je takova ze nejprve se ucim jak se delaji subtasky a pak az je budu umet, tak se budu vic zabyvat tim jak by bylo nejlepsi abych to delal s kuchary v populaci.

    \item KL div reward vuci populaci zatim beru jako minimum ze vsech kl div. TODO: rozmyslet jestli by nebylo lepsi pocitat prumer pres vsechny.
        Je potreba taky se zamyslet nad koeficientem tohoto rewardu a asi nejspise i jeho clipingem. Tj. aby byl dost silny na to aby mel dopad, ale zaroven hlidat jeho meze, aby se stale resil puvodni problem.

    \item Rozmyslet jako roli hraje KL div vuci entropii. Jestli v KL Div uz neni entropie zachycena. Proto nejspis pak pro modifikaci loss nechat pouze cross entropii.

    \item S kazdym dalsim jedincem v populaci prodlujuzu trenink dalsiho o 1e5 kroku
    \item Pro kazdou mapu jsem predtrenoval 30 SP agentu, ktere planuju vyuzit jako eval mnozinu.
    \item Prvni jedinec v populacnim treninku je SP, otazka jestli nepouzivat pro jednu mapu vzdy toho stejneho, aby se v experimentech zacinalo ze stejne strategie. Pak by se mozna lepe porovnavalo jak moc se dokazala populace zdiverzifikovat.
    \item TODO: should population partners play argmax during training?
\end{list}

\subsection{Summary 2.2.2023}
\begin{list}{}{}
    \item Vypada to ze jsem zprovoznil zakladni skript s gpu na metacentru. Nechavam nyni dobehnout 2 No reward populace, jedna se stoch policy partneru, druha s deter. 
    Po vyhodnoceni vuci eval mnozine (opet v kombinaci stoch i det) rozhodnu jakou cestou pujdu a pro tu spustim populace s KL\_div reward
    \item Asi by stalo za to vyzkouset i cross\_entropy dopad a pripravit scripty
    \item Zjistit jestli eval set s 0 neznamena stuck, pripadne zamyslet se nad anitstucking mechanismem
\end{list}

\subsection{Summary 3.2.2023}
\begin{list}{}{}
    \item Na metacentru se dopočítali dvě populace (15 jedinců), které byli inicializovány pomocí 5 SP agentů. Populace se liší v tom, jestli partneři při samplování používali argmax nebo ne.
    Při evaluaci během tréninku stále používají deterministic. Toto si myslím, že je stále dobrý nápad protože pokud jsou dobří vůči argmax, tak by neměli být špatní při STOCH, ale obráceně to neplatí.
    Popualce se jmenují POP_SP_INIT_ROP0.0_MPOP_BRCoef0.0_BRClip0.0_LCoef0.0_LClip0.0_DSRTrue_PADFalse a POP_SP_INIT_ROP0.0_MPOP_BRCoef0.0_BRClip0.0_LCoef0.0_LClip0.0_DSRTrue_PADTrue
    \item provádím eval vyhodnocí vůči SP eval množině. Zajímají mě výsledky při DETERMINISTIC a STOCH vyhodnocení. Zajímavé by asi mohlo být i kdyby jen jedna strana byla DETERMINISTIC, možná později.
     
    PADTrue
    mean SP part: 40.58
    mean SP of non zeros: 112.72222222222223
    zero SP ratio: 0.64%

    mean POP part: 67.35
    mean POP of non zeros: 83.14814814814815
    zero POP ratio: 0.19%

    PADFalse
    mean SP part: 23.06
    mean SP of non zeros: 98.82857142857142
    zero SP ratio: 0.7666666666666667%

    mean POP part: 60.86666666666667
    mean POP of non zeros: 80.79646017699115
    zero POP ratio: 0.24666666666666667%

\end{list}



\end{document}
